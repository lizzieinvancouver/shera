\documentclass[12pt,a4paper,oneside]{article}
\renewcommand{\baselinestretch}{1.8}
% \renewcommand*{\thefootnote}{\fnsymbol{footnote}}
\usepackage{sectsty,setspace} 
\usepackage[top=1.00in, bottom=1.0in, left=1in, right=1in]{geometry} 
\usepackage{graphicx}
\usepackage{epstopdf}
\usepackage{amsmath,latexsym,amssymb,wasysym}
\usepackage{natbib}
\usepackage{textcomp}
\usepackage{wrapfig}
\usepackage{hyperref}
\usepackage{float}
\restylefloat{table}
\usepackage[table]{xcolor} % http://ctan.org/pkg/xcolor

\setlength\parindent{0pt} % no indents throughout

\parskip=5pt
\pagenumbering{arabic}
\pagestyle{plain}
% squeeze space
\linespread{0.99}
\addtolength{\itemsep}{-0.05in}
\usepackage{multicol}
 

\newenvironment{smitemize}{
\begin{itemize}
  \setlength{\itemsep}{1pt}
  \setlength{\parskip}{0pt}
  \setlength{\parsep}{0pt}}
{\end{itemize}
}


\usepackage{fancyhdr}
\pagestyle{fancy}
\fancyhead[LO]{CAREER 2015}
\fancyhead[RO]{E. M. Wolkovich}

\newcommand{\Section}[1]{\vspace{-8pt}\section{\hskip -1em.~~#1}\vspace{-3pt}} 


%%%%%%%%%%%%%%%%
%%% START HERE! %%%
%%%%%%%%%%%%%%%%


%%%%%%%%%%%%%%%%%%%
%% To do notes %%
%%%%%%%%%%%%%%%%%%%
% class is Analytical methods in modern temporal ecology

%%%% Need to add %%%%
% (1) Broader impacts -- Outreach supports me; mention recent work in classroom and mentoring, improving those skills and moving further into new environments (public etc.) 
% (2) Integration of teaching/research needs work (note that this is not a broader impacts, don’t put it there).
% (Mayvbe?) The storage effect is \emph{not} neutral---species coexist via blah, blah---but it does allow coexistence of species which would appear identical if observed outside of germination rates across time (short snapshots etc.).

\begin{document}
\bibliographystyle{/Users/Lizzie/Documents/EndnoteRelated/Bibtex/styles/amnat}
\renewcommand{\labelitemi}{$\textendash$}
\iffalse
Concerns about the {\bf research} centered around four major points:
\begin{enumerate}
\item Research and teaching were not fully integrated. \emph{Response:} (1) Undergraduate education and training now included to address data gaps in niche/neutral section via partnership with RMBL REU program and Harvard grants to support summer reseach at RMBL. (2) Summer course now integrates long-term datasets and uses students' efforts to help determine best statistical models for end, peak and length of flowering (see niche/neutral section).
\item First part of proposal on field/greenhouse work was not sufficiently novel. \emph{Response:} Have removed the whole section.
\item Analyses proposed for niche/neutral would not provide desired inference/conceptual framework inappropriate. \emph{Response:} Have greatly extended this section to provide clarity on exact aims; now propose multiple, complementary methods to address question of interest and show clear links between aims and methods.
\item Detail on storage effect modeling insufficient. \emph{Response:} Have greatly extended this section, now providing model equations and more detailed methods of approach and analyses.
\end{enumerate}

Also, there were concerns over synthesis of the three research components.\\

The {\bf broader impacts} were generally considered excellent (Lilacs at Arboretum) but needed better integration (see above) and concerns were raised that they were `overly ambitious,' including especially over time needed to run 3-week summer course. \emph{Response:} Have piloted citizen science work in lab this year, it's going well. Have gained support of six colleagues to co-teach course and gained one term of teaching release to prep and teach first version of course. 

\newpage
\fi
{\bf Project description for CAREER: The Role of time in structuring Anthropocene plant communities}
\vspace{1ex}

\emph{Responses to previous feedback:} This proposal is greatly improved by feedback on a related submission last year. Based on reviewer comments I have removed a major research component centered around field and lab experiments to assess effects of phenology on fitness. This has allowed me to greatly expand on the approach and methods for the two remaining components---which reviewers were excited about---and will allow me valuable additional field time in Gothic, Colorado. I have also worked to better integrate across my research and educational components and clearly show how each feedbacks into the other. I have now added undergraduate education and training to an additional major research component and re-designed my proposed three week Statistical Institute to leverage students' help in developing local climate data and determining the best climate data for the two proposed research components. Finally, I have addressed concerns that my educational goals may be overly ambitious by: (1) leveraging support from colleagues to help teach the Statistical Institute and teaching and assessment resources at the Bok Center for Teaching and Learning, (2) switching the first offering of the Statistical Institute to January which will move it out of my field season and allow time to prep for it in the fall, (3) piloting a citizen science program for collecting phenological data at the Arnold Arboretum this year to show feasibility. \\

\begin{center}
{\sc I. Career development plan}
\end{center}

{\bf A. Overview, driving questions \& goals}
\vspace{1.5ex}\\
Thirty years ago a transformation in ecology was underway. Spurred in part by increasing rates of anthropogenic habitat change and loss, ecologists were changing their plot sizes \citep[][]{Soule:1992pw,Terborgh:2001bw}, their statistical methods \citep{legendre1993} and questioning many of their fundamental concepts and theories. The outcome of this transformation was the rise of spatial ecology as a major, cross-disciplinary field that addressed both basic and applied questions with new vigor, methods and theories, and fundamentally reshaped ecology's view of the role of space in structuring populations and communities \citep{Doak:1989oc,Turner1989}.
\vspace{1.5ex}\\
Questions of how altered habitat space affected communities occupied me as I started my research career, working in the fragmented system of San Diego, California. Since then, however, my focus has shifted to questions of time, how it structures ecological systems, and how human modification of the earth system---the hallmark of the epoch known as the Anthropocene---has altered it.
\vspace{1.5ex}\\
I argue that recent climate change \citep[at least partially associated with increases in anthropogenic greenhouse gases, see][]{Trenberth:2007hk} has produced dramatic shifts in how organisms experience time. Anthropogenic climate change represents a fundamental transition in most ecosystems from a long period of relative stationarity in climate to a period of dramatic nonstationarity (Figure 1). Across many ecosystems nonstationarity includes lengthening growing seasons and altered climatic variability, with such changes projected only to increase in the future \citep{knutti2013}. Climate change has created a pressing need for ecology to better understand how time---both within and across years---shapes ecological systems. My research program is anchored by this need and guided by the critical question: \emph{How do communities dis-assemble and assemble due to the shifting temporal dimensions of climate change?} I address this question through the lens of plant phenology---the timing of recurring life history events---and the most reported biological indicator of climate change.
\vspace{1.5ex}\\
Understanding how temporal plant community assembly is affected by climate change requires integrating direct impacts on phenology due to climate change with new approaches in coexistence theory to accommodate non-stationarity. \emph{In the next five years I propose to make major strides in this area by:}
\begin{enumerate}
\item Testing how climate change alters the prevalence of niche versus random temporal assembly by developing new methods to study shifting patterns of phenological overlap in a 38-year community flowering dataset of alpine meadow species. % How does climate change alter the phenological assembly of flowering plant communities? 
\item Examining how non-stationarity in the climate impacts long-term coexistence mechanisms by adapting the storage effect model to include phenology and nonstationarity in the environment.
\end{enumerate}
\begin{center}
\includegraphics[width=0.9\textwidth]{/Users/Lizzie/Documents/git/grants/career/2015/figures/hockeystickfull.png}
\end{center}
These research goals are complemented by major educational goals that leverage the tangible and personal nature of plant phenology and its related wealth of long-term records. Phenology is the most reported biological indicator of climate change, which I believe is because it is tied inherently to how humans mark the passage of time. It thus benefits from records covering thousands of species and hundreds of years \citep{Chuine:2004fk,tansley}. Such records highlight the impact of anthropogenic climate change, but also the difficulties of working with time-series data, especially when such data contain nonstationary periods as seen with climate change. I believe many ecologists are undertrained in how to handle such data---which most often violates assumptions of independence and stationarity. \emph{Elevating temporal ecology, thus, requires a fundamental advance in ecological theory, as proposed above, and in how students and the public alike understand time-series data. In my educational components I thus propose to:}
\begin{enumerate}
\itemsep 0em
\item Train a new generation of ecologists in time-series methods that are applicable to both fundamental ecological research and climate change research (with a focus on autocorrelation, scaling and nonstationarity)
\item Bring this fundamental understanding of nonstationarity seen with climate change, which is effectively a way to emphasize how there can be both variability but also a strong---and important---trend over time, to the public. 
\end{enumerate}
\vspace{1ex}
{\bf B. Results from prior NSF support: Postdoctoral Award \# 0905806}
\vspace{1.5ex}\\
My research career is focused on understanding community assembly through the lens of plant phenology. To date I have been struck by a disconnect in the amount of data we have on plant phenology and our understanding of findings from those data. I believe this disconnect is caused by a fundamental lack of understanding of how plant phenology affects and is itself shaped by community structure \citep{tansley}. In my career I have worked to address this by first documenting the variation and then testing for evidence that phenology is an important force structuring plant communities using the special case of plant invasions. This work was supported by a Postdoctoral Research Fellowship in Biology (Award \# 0905806) that I received in 2009. I review below the two major research questions I addressed with this grant, which resulted in 14 papers on which I was an author---7 of which I was first author. \\
\vspace{-1ex}\\
\noindent \emph{Question 1. What are the direct phenological effects of climate change on plant communities?}\\
\vspace{-1ex}\\
\indent To answer this question  I tested how climate (temperature, precipitation, irradiance) alters plant phenology and which attributes of species and habitat predict plant responses. In collaboration with an NCEAS working group that I led with Ben Cook (climatologist, NASA GISS and Lamont-Doherty), I conceptualized, collected and compiled phenological data on \(>\)5,000 plant species with \(>\)30 years of data on approximately 2,000 of those species. This is by far the most species-rich phenological database available. Combined with climate data and the phylogenetic tree we built encapsulating their evolutionary relationships \citep{phentree} it is a powerful resource for testing how plants respond to climate change.
\vspace{1.5ex}\\
Analyses from this database revealed compelling patterns that have formed the foundation of my proposed research. Across habitats, many species exhibit a wide but predictable set of responses to temperature and precipitation \citep{Cook:2012,pauncc}, sensitivity to temperature is greatest in northern hemisphere communities \citep{Cook:2012}, and for early-flowering species \citep{Wolkovich:2012n} and approximately \(20\%\) of temperate species have chilling requirements that delay flowering with global warming \citep{Cook:2012pnas}. Additionally, I led the first large-scale effort to compare phenological responses in warming experiments to observational trends \citep{Wolkovich:2012n}. Contrary to expectations, I found that experiments underpredicted plant responses to temperature; a finding consistent across studies of different designs, durations and when considering species for which we had both experimental and observational data. Such differences between observed and experimentally-estimated responses to temperature highlight the need for better theory and models in the area of phenological research.  \\

\vspace{-1ex}
\noindent \emph{Question 2. How does phenological assembly in an era of changing climate contribute to plant invasions?}\\
\vspace{-1ex}\\
\noindent Understanding the direct effects of climate on phenology provides only the first piece of the puzzle in predicting community shifts with climate change: while species may directly respond to climate, they may also respond to the temporal shuffling of other species. As an initial approach to test for such indirect effects I used the special case of plant invasions. I hypothesized that exotic species can invade and increase in their introduced communities because they (a) occupy vacant temporal niches---especially if climate change has created `novel niche space' at the start and end of the growing season---or (b) because they are more flexible in their phenological response to climate \citep{wolkovich:2010fee}. My research, using long-term community datasets, found that exotic species in some habitats have distinctly earlier phenologies compared to their introduced communities and across habitats appear to be more flexible in their phenologies than native species \citep{wolkovich2014aob}. \\
\begin{center}
{\sc II. Project components: Integrated research \& educational goals}
\vspace{0.5ex}\\
\end{center}
{\bf A. Overview}
\vspace{0.5ex}\\
Community assembly in ecology today is anchored by two extremes. On one side is the Hutchinson niche world \citep{Hutchinson:1959xi}, where a community of species---each with its own $n$-dimensional niche---may coexist indefinitely. On the other side is neutral assembly \citep{Watterson:1974gk,Caswell:1976np}---where species with identical niches may co-occur for long periods of time until demographic stochasticity yields the rise of one species. Under the neutral model, communities are a random assortment of species that depend on dispersal into a site and demographic stochasticity---thus the resulting communities should appear random with respect to their traits as opposed to carefully ordered, as predicted by niche theory. One version of the neutral model, neutral biodiversity theory \citep{Hubbell:2001vo}, has received particular attention of late \citep[note that I use the term `neutral' and `random' to refer to any model where species' traits do not impact long-term co-occurrence, \emph{sensu}][]{Kraft:2008oz} and spurred many studies on `niche' versus `neutral' communities. The end result has been the realization that nature rarely appears to settle upon such extremes and most communities exist, instead, in the spectrum in between \citep{Chase:2007yt,Vergnon:2009bh}. % ADD MORE recent citations here?
\vspace{1.5ex}\\
Both these extremes, and intermediate positions in between, however, assume the environment allowing long-term species co-existence or co-occurrence does not shift dramatically over short timespans (e.g., several years or decades). Such environmental shifts could easily lead to the favoring or dis-favoring of some species over others in a niche-world or precipitate a rapid rate increase in a neutral world---such that some species' slow walk to extinction could accelerate to a quick jog. \emph{As climate change is defined by its rapid environmental change, a driving question is: how does climate change alter coexistence?}
\vspace{1.5ex}\\
Addressing this question requires overcoming two major limitations of current coexistence theory: its emphasis on space, and its assumptions of generally stationary environments. While coexistence theory is inherently temporal---as it is the long-term nature of species co-occurrence the theory aims to explain---much recent work has focused on space, from predicting and explaining spatial patterns  to using space to promote coexistence \citep{gilbert2004}. In contrast many of the changes organisms experience with climate change involve altered temporal dimensions: longer growing seasons or altered rainfall patterns, for example. 
\vspace{1.5ex}\\ 
These changes represent an important temporal nonstationarity rarely considered in coexistence theory, as they effectively stretch or shorten the amount of temporal niche space available---temporal space which has been relatively constant in most systems over the last 11,000 or more years of community assembly \citep{Trenberth:2007hk}. Alongside these warped temporal spaces are additional shifts in climate fluctuations---more frequent or extreme rare events such as major droughts, cold snaps or heat waves. While some coexistence theory is built on environmental fluctuations \citep{Chesson:1997dz}, these shifts still represent important departures from the current realm of ecological theory: most theory to date assumes the environment is stationary (see Fig. 1), and thus assume that while the environment may fluctuate from day to day or year year, the underlying distribution that describes the fluctuations does not shift. 
%JD: This gets a bit confusing because you talk about shifts in climate fluctuations as well as environmental fluctuations, and then  refer to something (the environment) that may fluctuate and the underlying distribution that describes these fluctuations.... Perhaps make it clear where the non-stationarity comes in (and maybe note that whilst some theory might consider fluctuating environments, none addresses non stationarity in these fluctuations - not sure if this is true ...). 
\vspace{1.5ex}\\
Addressing these limitations requires research with a distinctly temporal approach that explicitly examines the impacts of nonstationary environments. Research in plant phenology is particularly promising to take on this challenge as it is supported by a suite of previous work on the temporal nature of community assembly via phenology \citep{Gleeson:1981wh,Rathcke:1988yc}, and is an area where the nonstationarity of climate in recent decades has had clear, documented impacts \citep{Menzel:2006sq}. Efforts to document these impacts have also highlighted the limitations of simply predicting phenological shifts via shifts in the abiotic environment and suggested other factors---especially biotic interactions---may underlie many species' responses to climate change \citep{Pau:2011wd}. 
\vspace{1.5ex}\\
My proposed work will combine the common climatological focus on understanding phenological shifts with inferences from coexistence theory. In two complementary areas I will advance coexistence theory to more explicitly consider phenological assembly and nonstationarity. First, I will re-approach niche-neutral theory to examine assembly of a flowering plant community both within years---examining phenological assembly within a growing season---and across years---looking at shifts in phenological assembly alongside shifting climate. This work will provide fundamental insight on how phenology affects community assembly via temporal niche space while addressing how such assembly may have shifted over recent decades in one system (Gothic, Colorado). Second, to understand how assembly may shift over longer timescales and in other habitats I will adapt a fundamental model of community assembly to nonstationary environments. This model integrates several mechanisms for coexistence, including one directly dependent on a fluctuating environment \citep{Chesson:1997dz} and will allow tests of how dominant coexistence mechanisms change as the environment shifts from stationary to nonstationary. \\
\begin{figure}[h!]
\begin{center}
\includegraphics[width=0.5\textwidth]{/Users/Lizzie/Documents/git/grants/career/2015/figures/figs3.png}
\end{center}
\end{figure}
\vspace{1.5ex}\\
{\bf B. Phenological niches in shifting environments}
\vspace{1.5ex}\\
\emph{Introduction:} Decades of research have documented shifts in plant phenology with climate change across the globe, but have also continually noted significant unexplained variation---suggesting phenology may be shaped by factors beyond the abiotic environment \citep{Pau:2011wd}. Ecological theory suggests phenology should be a major component of plant species' temporal niche \citep{gotelli1996}. Through the temporal niche model total niche space---which is limited in many habitats by abiotic forces (e.g., frost) that define growing season length---is divided such that species limit interspecific competition by avoiding phenological overlap \citep{Pau:2011wd,tansley}. Trophic factors are expected to play an important role, by controlling the optimal time for each species for pollination \citep{Brody:1997ro}, for example. Under this model, climate change would alter phenological niches by stretching and distorting the abiotic environment but the ultimate prediction of the expressed phenological niche for each species would depend also on its surrounding community---in short, it would depend on direct climate effects and indirect assembly effects.
\vspace{1.5ex}\\
Predicting phenological assembly is not a new concept. Thirty years ago, following on MacArthur's formalization of the broken-stick model \citep{MACARTHUR:1957gf}, a community assembly model based on random breaks of niche space, ecologists studied the pattern of flowering times of species in a number of communities. They questioned whether patterns indicated phenological niches, their timing carefully shaped by natural selection to minimize overlap, or simply randomness \citep{poole1979,Cole:1981il}. The debate featured prominently in ecology for some years \citep{Gleeson:1981wh,Fleming:1984hu}, yet it quieted without any clear resolution and the study of flowering times moved away from understanding how phenology may affect community assembly. 
\vspace{1.5ex}\\
Advances in ecological theory provide the opportunity to revisit the debate in a new light, while the rise of climate change offers new data and new urgency. Current research suggests most communities operate somewhere between the extremes of niche versus neutral assembly  \citep{Chase:2007yt,ellwood2009} and suggest the environment may be key in determining the relative contribution of niche versus random forces \citep{chase2011}. Thus, shifts in the environment with climate change may shift the degree to which communities are determined by these two processes.
\vspace{1.5ex}\\
I propose to test for evidence of shifting phenological assembly with climate change using a long-term community flowering phenology dataset from Gothic, Colorado \citep{Inouye:2008gj,CaraDonna2014}. Using two null models---one based on a mid-domain model \citep{Morales:2005ex} and the other on sorting from a regional species pool \citep{poole1979}---I will test for the apparent roles of niche versus random assembly across the past 40 years. While this work is focused on one particular community the methods developed will be applicable across communities to examine shifts in phenology with climate change---integrating both impacts from the abiotic and biotic environment.
\vspace{1.5ex}\\
\emph{Predictions:}
Addressing the question of how assembly may have shifted with climate change requires first considering what factors shape flowering phenology under a stationary environment. Critical for predictions from niche theory is the idea of limiting similarity \citep{HUTCHINSON:1959mz,ABRAMS:1980dq}: species co-occurring must diverge in at least some of their traits to prevent competition for the same niche space. Thus when considering any one trait it is important to consider additional traits that delimit which species may actually compete via the trait of interest---in this case phenology. As flowering is a critical component of pollination \citep{Brody:1997ro,botes2008} a major trait axis to consider is how species are pollinated (wind or animal) and for animal (zoophilous)-pollinated species, which plant species share the same pollinators. Much ecological research has focused on predictions of phenology from pollination mode resulting in two major predictions for how flowering times within a community of zoophilous species should be distributed:\\
\emph{Prediction 1a --} If plants compete for pollinators then limiting similarity theory predicts flowering times must diverge to prevent competition for the same niche space. Thus species that share pollinators would show overdispersion.\\
\emph{Prediction 1b --} Alternatively, if plants do not compete for pollinators and instead large multispecies floral displays increase pollinator visiting, then coflowering is expected. Thus species that share pollinators would show underdispersion.
\vspace{1.5ex}\\
Additional traits, however, may play a role in shaping flowering phenology. Seed size may interact strongly with flowering time---as large seeds require sufficient time to grow each season and cannot easily be produced by species that flower at the end of the growing season \citep{Mazer:1989in}. Second, research suggests a fundamental trade-off where species with a trait syndrome of rapid return on investment traits (e.g. lower leaf mass area, lower C:N etc.) may also flower early while the reverse trait syndrome is predicted for species flowering later \citep{Aldridge:2011,Craine:2012kl,tansley}. Additionally, for systems with a significant drought period some species may only flower before or after the drought, depending on their drought tolerance \citep{Craine:2012kl}. If phenology is constrained by these other traits than it follows that: \\
% this seems like it could be a prediction 2b - flowering phenology aggregated, but location of aggregation in growing season might vary year to year (which is why you can't aggregate data across years). 
\emph{Prediction 2a --} Flowering phenology will be strongly correlated with other major plant traits and will show patterns of dispersion that match to correlated trait syndromes (which themselves might be shaped by both neutral and niche processes). \emph{Prediction 2b --} If present, location of aggregation within growing season will vary year to year with climate. 
\vspace{1.5ex}\\
Alternatively, if phenology is not tied strongly to other traits nor under selection to maximize pollinator access it may drift with evolution \citep{Lechowicz:1984cr} or be structured simply by stochastic forces. Under these two scenarios:\\
\emph{Prediction 3 --} Over or underdisperion of flowering phenology would be predicted by the phylogenetic relatedness of the particular community \citep{davies2012eco}.\\
Note that this prediction assumes phylogenetic conservatism of flowering phenology, which to date has been found to be weak \citep{davies2013}, especially in the proposed community of study  \citep{davies2013,CaraDonna2015}.\\
\emph{Prediction 4 --} Flowering phenology would be random with respect to its distribution across the growing season. \\
Next comes the question of how climate change may impact phenological assembly. Recent work suggests the environment may drive the degree to which communities are structured by niche versus neutral forces. In one example, communities in `harsh' environments appear governed more by niche forces after some time \citep{Chase:2007yt}, thus predictions with climate change would depend upon the exact local nature of climate change and the community. However, predictions based on niche theory (via pollinators or other related traits, predictions 1-2) implicitly assume some level of equilibrium conditions---as they assume phenology has been shaped by long-term dynamics of pollinators and the environment. Thus, I predict that:\\
\emph{Prediction 5 --} Climate change will shift phenological assembly, however exact predictions depend on the degree to which predictions 1-4 are supported under average climate dynamics. For example, if species are partitioned into temporal niches via pollinators or show evidence of co-flowering climate change may shift dynamics to more random assembly. Alternatively, if assembly appears phylogentically patterned there may be little or no impact given no major shifts in species composition. If assembly is shaped by other traits the impact will be determined by how climate change impacts those traits (e.g., if climate change increases drought and favors drought-resistant species). If such shifts are seen, they will provide major insight into how the environment structures community phenology.
% Perhaps looking at the shift in phenological assembly with climate change might then also provide an indication of the forces structuring community phenology!
\vspace{1.5ex}\\
\emph{Methods:} 
I propose to test predictions using David Inouye's 38 year dataset (1974-2014, no data collected in 1978 or 1990) of flowering phenology from the Rocky Mountain Biological Lab in Gothic, Colorado. The dataset records flowering phenology in 2 x 2 m plots distributed across a variety of habitats including date of observations (with observations generally every other day throughout the growing season) and total number of flowers or flowering inflorescences. Number of individuals is not currently available as part of the proposed work I will work with programmers to attempt to extract this information which is generally embedded in the original Excel workbook cells' formulas. For this analysis I plan to focus on the `Rocky Meadow' and `Wet Meadow' plots as they represent the longest term data and two distinct habitats. 
\vspace{1.5ex}\\
Climate data include long term precipitation and temperature data from the local Crested Butte National Oceanic and Atmosphere Administration weather station and measure of snowmelt date each year noted each year by Billy Barr. Working with students from the Summer Institute (see section III below) I also plan to create back-projected weather data for the `Rocky Meadow' and `Wet Meadow' areas by using data from Crested Butte and plot-level data collected by David Inouye over several years. 
\vspace{1.5ex}\\
Pollinator and other additional trait data will be collected through two means. To assign pollinators to plant species I plan to draw on available data on bee species local to the Inouye plots from Dr. Becky Irwin who has been collecting data for five years and Dr. Berry Brosi who has been collecting data on bees and their plant networks at RMBL. Next, over two summers a postdoctoral fellow and several undergraduate students will collect pollinator data from `Rocky Meadow' and `Wet Meadow' plots including overall pollinator estimated visitation rates and identification of pollinators (when possible) via catch and release. These researchers will also collect needed data on seed mass, metrics of tissue return on investment \citep{handbook2013} in the `Rocky Meadow' and `Wet Meadow' plots. We plan to collect data from at least 10 individuals of each species adjacent but not in plots with species collected from both habitats if present in both. 
\vspace{1.5ex}\\
To test predictions for the dispersion of flowering phenology I will build null models. Traditional null model methods often use mean trait values (e.g. see Fig 2, top), while in contrast theory considers the full trait space occupied. The Inouye dataset is very rare among long-term phenology datasets because it includes data on number of flowers produced every other day, this should allow estimates of species' full phenological niches year to year. Total overlaps among these curves (see Fig. 2) will then be estimated for each plot and compared to predictions from two overarching null models.
\vspace{1.5ex}\\
I propose two null models that test for community dispersion (1) given the resident community of a plot and (2) given the regional species pool. While many null models are possible for these data I believe these two bracket the two most important extremes. The first null model is akin to a mid-domain model of the growing season \citep{Morales:2005ex}. This model will thus test if communities appear random or overdispersed at the plot level given the established species composition. The next null model will examine community dispersion compared to the regional species pool \citep{poole1979}. For this I will consider all species from both the `Rocky Meadow' and `Wet Meadow' areas as the regional species pool. The phenological curve for each species each year to be used in the null model will be calculated by (1) first calculating the curve for each species within each habitat (across all plots), then (2) averaging the two curves for species present across both habitats. With these curves in hand each plot's null model will be determined each year by repeated random draws from the regional species pool based on the total number of species present per plot. These two overarching models will be adjusted to test specific predictions: for example, they may be repeated using only species that share pollinators and I will control for varying numbers of plots across years and habitats.
\vspace{1.5ex}\\
Analyzing results of the null model tests will lead to several statistical challenges, but also the opportunity to refine null models and develop our methods to consider them in their increasing complexity. It will require a high number of tests and thus care and continual re-evaulation of the best approach to robustly comparing the full suite of predictions will be taken. Currently, I plan to consider plots as replicates within habitats, though I will also examine variance among plots within habitats. I plan to use habitat and year as the main predictor variables---and with support of students in the Statistical Institute will narrow climate variables to 2-3 that best describe variation across space and time, including possible important lag effects. I plan to use resampling and Monte-Carlo methods to examine effects. An additional challenge may be applying null models to separate suites of species that share pollinators---as some sets of species with a shared pollinator may be quite small. I hope the measurement of full phenological curve overlap (versus distances between simple means) and the plot-level replication will provide some statistical leverage for this issue but I also plan to look for creative approaches. This is a fundamental issue in assessing limiting similarity as at the appropriate level each species shoud be alone with its particular set of traits thus tackling this challenge is an important part of the project.
\begin{figure}[h!]
\begin{center}
\includegraphics[width=0.7\textwidth]{/Users/Lizzie/Documents/git/grants/career/2015/figures/RMBL_rm8.png}
\end{center}
\end{figure}
\vspace{1.5ex}\\
\emph {Links to educational goals:} This project will benefit from support by one postdoctoral fellow, 3-4 undergraduate researchers and the efforts of all the students of the Statistical Institute. The postdoctoral fellow will work for 28 months that will encompass three summer field seasons in RMBL. The first year will be devoted to training on the plants and pollinators and pilot work; undergraduate researchers will join the following year and data collection will begin in earnest. I will work closely with all these researchers to define the full scope of the project and will travel with them to Gothic, Colorado to work on the project with collaborator Dr. Inouye. I will assess success of the undergraduate researchers' training at the end of each summer via The Survey of Undergraduate Research Experiences (SURE III). Students from the Statistical Institute will (1) develop methods to back-estimate habitat-level climate data and (2) will help narrow the list of climatic factors to 2-3 primary, possible aggregate, measures that best describe variation across years and habitats.
\vspace{1.5ex}\\
\emph{Preliminary results:} Preliminary results using a simple method of examining variance of the spacing between peak flowering dates show overall evidence of weak overdispersion in `Rocky Meadow' (standardized variance mean $\pm$ SD: $-0.13 \pm 0.75$)  and `Wet Meadow' ($-0.20 \pm 0.85$) but high variability across years and habitats (see Fig. X for one example). This variability appears possibly related to shifts in climate (Fig. X) and preliminary analyses suggest communities may shift from moderate overdispersion and towards more random or clumped dispersion in extreme climate years. 
\vspace{1.5ex}\\
\emph{Expected outcomes \& significance:}
Understanding how niche and neutral processes may structure the phenological diversity of plant communities---and how these processes may shift with climate change---has profound implications for predicting future community and ecosystem responses, as well as fundamental repercussions for basic ecology. The proposed work will test a basic premise of the role of phenology in structuring a plant's temporal niche but will also test whether climate change fundamentally disrupts any niche structuring. Given these results it should be possible to determine how well we may be able to predict future plant communities at RMBL under continuing climate change. Further, the new methods developed and applied to the RMBL plant community should be applicable to other systems and hopefully spur more studies of how assembly itself is altered by climate change. 
\vspace{1.5ex}\\
{\bf C. A storage effect model for the Anthropocene}
\vspace{1.5ex}\\
\emph{Introduction:} Predicting community shifts with climate change requires fundamental appreciation of the diverse mechanisms that govern how communities assemble---within any community varying forces may be at play for different species or subsets of species \citep{macarthur1958}. Thus a complete view of community assembly through the lens of phenology and climate change requires consideration, integration and development of several major models. My above-work will examine two major mechanisms: neutrality and niche, with a focus on testing for evidence of within-year temporal niches via phenology. A species' temporal niche, however, may also occur across much longer timescales. This is considered in another major mechanism by which species may co-exist---one with an intrinsically temporal focus---the storage effect \citep{Chesson:1997dz}. Under this model highly similar species coexist via small differences in how they respond to temporal variability in the environment.
\vspace{1.5ex}\\
\emph{The storage effect model has found increasing support in recent years \citep{Angert:2009,Levine:2009ym}, suggesting it may be an important mechanism underlying how communities assemble. The model, however, rests on one major assumption that is violated in most environments today: it assumes a stationary environment.} Climate change has shifted most systems to nonstationary climate dynamics \citep{ipcc2013,knutti2013}. \emph{Thus, understanding how communities may assemble via the storage effect requires extending the model to nonstationary environments}, then examining how communities assemble under stationary and nonstationary environments. Importantly, as climate in many regions of the globe was relatively stationary for thousands of years until recent climate change \citep{ipcc2013}, such an extension of the model allows testing how communities built on coexistence via variable environments respond when the environment switches to nonstationary. 
\vspace{1.5ex}\\
Considering the storage effect model from the lens of phenology and nonstationary environments highlights one additional assumption of all current implementations of the model: that species do not adjust their phenology with the environment. In contrast to this much research highlights that many species shift their timings from year to year with warmer springs \citep{tansley} or shifting rainfall \citep{Crimmins:2011lq}. Such `phenological tracking' may be critical to predicting population shifts with climate change as recent empirical studies have found that the species that phenologically track climate change most closely also appear to  increase in abundance \citep{Cleland:2012vn,wolkovichAmBot2013}, while species that do not phenologically shift with climate change may go locally extinct \citep{Willis:2008bf}. \emph{Yet there is little current theory to test under what environmental regimes phenological tracking may be favored. Thus, this model will provide a major way to test the role of phenological tracking in assembling plant communities}, by allowing studies of how a species' level of phenological tracking impacts its coexistence and abundance under varying environmental regimes. 
\vspace{1.5ex}\\
\emph{Model description:} 
I propose a version of the storage effect model that merges within-year dynamics---thus allowing integration of phenological tracking into the model---with between-year dynamics, thus allowing studies of how coexistence mechanisms shift in nonstationary environments. Consider a community of $N$ species where each species has a long-lived life history stage modeled here as a seedbank, $s$. Each species' long-term population dynamics as time progresses ($t+1$) are composed of two parts: recruitment and survivorship. Considering one species' ($i$) recruitment: seeds germinate from their seedbank each year at a rate $g_{i}$. This rate is described as a gaussian curve with peak $g_{max}$, width $h$ and centered at $\tau_{i}$.  The germination in any particular year depends on the environment that year: in particular when the growing season starts ($\tau_p$) in comparison to the optimal timing for that species ($\tau_i$). After germination the total number of new seeds contributed to the seedbank depends upon a species per-germinant fecundity in the absence of competition ($\lambda_i$) and competition with the surrounding community, which is determined by the total intra- and interspecific competition, described via Lotka-Volterra competition. Survivorship of non-germinating seeds in the seedbank is determined by a simple loss rate ($s_i$). Thus, considering a simplified community of two species $i$ and $j$ yields:
\begin{align}
 N_i(y+1) & = N_i(t)s_i + g_iN_i(t)\frac{\lambda_i}{(1+\alpha_{ij}g_jN_j(t)+\alpha_{ii}g_iN_i(t)) - s_i}
\end{align}
These equations follow closesly to previous work \citep{CHESSON:1994vn,Chesson:2004eo} with one major adaptation to allow some species to `track' the start of the growing season via flexibility in $\hat{\tau_i}$, given the following:
\begin{align}
&\hat{\tau_{i}} = \phi_i \tau_{p} + (1-\phi_i)\tau_{i}
\end{align}
Here, a single parameter, $\phi_i$, varies from 0 to 1 where $\phi_i=0$ yields a species with no tracking and $\phi_i=1$ yields a species which obtains its maximum germination each year as $\hat{\tau_i}=\tau_p$.
\vspace{1.5ex}\\
This model includes three mechanisms of coexistence: (1) the storage effect, (2) relative non-linearity, and (3) niche differences that result from variation in species-specific parameters (e.g., $\tau_i$, $\lambda_i$). The model is designed to include variation in the environment through shifts in the start of season ($\tau_p$), which is drawn each year from an underlying distribution. Thus, nonstationarity to the environment can be included by varying the underlying distribution of $\tau_p$ through time. 
\vspace{1.5ex}\\
\emph{Predictions:} 
I plan to consider two major types of nonstationarity by varying underlying distribution for the start of season (SOS): (1) a linear shift towards early SOS, and (2) increasing extreme events defined here as SOS two standard deviations from the mean. Considering a model where species do not track the SOS (no Equation 2) allows tests of how coexistence mechanisms shift as the environment changes from stationary to nonstationary---I predict that:\\
\emph{Prediction 1a --} A shift towards earlier springs will favor species with earlier optimal germination timings ($\tau_i$), and \emph{Prediction 1b --} will shift the prevalence of coexistence mechanisms towards favoring coexistence via niche dynamics, as this mechanism is less dependent on the environment compared to the storage effect and relatively non-linearity.\\
\emph{Prediction 2 --} A shift towards more extreme events may also shift the prevalence of coexistence mechanisms but the exact outcomes will depend upon the magnitude of the extreme events and how frequent they are compared to species' seedbank survival times.
\vspace{1.5ex}\\
Next, considering a model where the species may track the start of the season allows tests of how phenological tracking may alter community assembly. Considering a community where only one species tracks (or one tracks more closely than the rest) and all other parameters are constant across species yields an obvious first finding: the species that tracks will drive all other species extinct given a stationary or nonstationary environment because it will reach its maximum germination rate each year. Thus understanding how assembly is affected by phenological tracking is best framed as a question of trade-offs. Recent research shows that in many communities it is the early-season species that track climate most strongly \citep{Cook:2012} and these species often have traits associated with lower competitive abilities \citep{wolkovich2014aob}. Thus, I predict that:\\
\emph{Prediction 3 --} Maintaining species-rich communities where some species track will require trade-offs between tracking ($\phi_i$) and other parameters affecting a species' competitive ability (e.g., interspecific interaction coefficients $\alpha_{ij}$ and per germinant fecundity, $\lambda_i$). Larger trade-offs will be required in environments with larger degrees of nonstationarity. 
\vspace{1.5ex}\\
\begin{figure}[h!]
\begin{center}
\includegraphics[width=1\textwidth]{/Users/Lizzie/Documents/git/grants/career/2015/figures/taus_2panel.png}
\end{center}
\end{figure}
\vspace{1.5ex}\\
\emph{Methods:} 
I plan to test the model using a simulation approach, built around analytical solutions to assess coexistence (via calculating the long-term low density growth rate for each species as invader) and to measure each of the three mechanisms of coexistence. While this will not yield an analytical nonstationary solution to the model---which to date I have found not possible without severely limiting assumptions---I believe it provides a major step forward and the best approach to address my questions of interest. 
\vspace{1.5ex}\\
Nonstationary simulations will be guided by recent climate change and projected future changes \citep{ipcc2013}. I plan to look at a linear shift and an increase in extreme events---for each of these approaches I plan at least three levels of nonstationarity (low, medium, high), which I expect will be guided by varying emission scenarios. I next plan to take two major approaches to evaluating my predictions via simulations. First, I will perform a set of parameter sweeps where coexisting communities of $n$ species (probably 2, 10 and 100) are identified under a stationary environment and then nonstationary scenarios are applied. Second, given these results I plan to have targeted model runs that examine the effects of certain parameters highlighted by the parameter sweeps. In particular, $\tau_i$ will always be examined and for communities of phenological tracking I expect to also focus on tradeoffs with $\alpha_{ij}$ and $\lambda_i$. Full parameterization of the model is a complex challenge as the model requires linking data on the environment, germination rates and curves, and competition metrics. Thus, I plan to compile all available data to inform parameterization whenever possible and when not possible I will highlight these gaps and test a large range of parameter space to understand how such missing data impacts model outcomes. 
\vspace{1.5ex}\\
The model and methods represented here are, I believe, the best for our major questions after considering other models and approaches for modeling phenological tracking and environmental change. While I see many exciting additional questions that could be addressed with changes to this model (e.g., allowing species invasion or associating mortality with tracking in extreme years) I believe the currently outlined approach is the best starting point and I hope the freely-available code will allows others to tackle these questions.  
\vspace{1.5ex}\\
\emph {Links to educational goals:} This project will supported by the research efforts a postdoctoral fellow who will lead work on parameter estimates and simulations. Students in the Statistical Institute will be integral in developing best estimates for the nonstationary environment simulations through work with recent and projected climate data. If time allows in the second year of the Institute they may also use past climate reconstructions to examine how past climate change compares to current and project future change, and thus may have altered coexistence. 
\vspace{1.5ex}\\
I currently feature the concept of how temporal variability in the environment may produce coexistence in my major public outreach lecture (\emph{The Race for Spring: How climate change alters plant communities}). Using examples from several sites I introduce the concept of how species may coexist through time to produce diverse communities and build this into a discussion of climate variability and climate change. Thus I expect this project to be a major foundational part of my future public outreach talks, including a new series of public lectures I plan to develop where I explain both climate projections and introduce my results as a starting point for building ecological projections at the community-level.
\vspace{1.5ex}\\
\emph{Preliminary results:} As I propose two major new extensions of the storage effect model---extending the model to a nonstationary environment and adding phenological tracking---I have already begun work on this project to show proof of concept. In consultation with Dr. Donahue (see letter of support), I developed the basic suite of equations which can be adapted to $n$ species and have begun to examine the model with an environment that switches from stationary to nonstationary (Figure Y). Results suggest clear support for \emph{Prediction 1a} and preliminary support for \emph{Prediction 1b}, though we are continuing to refine our equations the three coexistence mechanisms. 
\vspace{1.5ex}\\
\emph{Expected outcomes \& significance:}  I believe this project will start a major shift in how ecologists think about coexistence via environmental variability---by introducing nonstationarity into models where stationarity has always been previously assumed. In addition this project will demonstrate and stress how to connect climate change projections to coexistence modeling. Finally, I expect this project will highlight a number of previously un- or under-recognized assumptions and highlight data needs to build models that may usefully project community shifts with climate change. Related to this, all \verb|R| code for the model and all climate change scenarios used for simulations will be made public via \verb|github.com|  to encourage other researchers to build upon the work. % While climate change represents one major and recent example of such nonstationarity, the paleorecord shows that many ecosystems have appeared nonstationary over timescales relevant to community assembly \citep[e.g.,][]{Jackson:2009el,cook2010}. I believe robust coexistence theory thus must embrace nonstationarity.
\vspace{1.5ex}\\
{\bf D. Links between two major research components}
\vspace{1.5ex}\\
The two major research components---(1) \emph{Phenological niches in shifting environments} and (2) \emph{A storage effect model for the Anthropocene}---represent distinct but highly complementary projects that will provide major new ways to examine coexistence in an era dominated by global change, where nonstationary environments are the norm, rather than the exception. While the first project provides fundamental concepts and methods to consider how phenological assembly operates under a shifting environment, the second project adapts a major assembly model for nonstationary environments. Taken together these new approaches will allow community ecologists to consider how coexistence has been altered already by climate change and project forward to the future. \\
\begin{center}
{\sc III. Project components: Cross-component educational goals}
\vspace{0.5ex}\\
\end{center}
{\bf A. Overview}
\vspace{1.5ex}\\
In addition to the educational goals embedded in the research projects described above I propose several major educational components that cut across the proposed projects and build on my desire to develop a new approach to temporal ecology and to educating the public on climate change. The proposed activities below also build on unique attributes of my home organization---the Arnold Arboretum---which provides special opportunities to run intensive short courses and to reach a diverse public audience. In addition I plan to leverage outreach opportunities through the Arboretum's regular programs: I and the postdoctoral researchers will lead tours at the Arboretum, which are often requested by local schools, colleges and groups, lead \href{http://arboretum.harvard.edu/visit/tree-mob/}{Tree Mobs} (this is a type of outdoor informal lecture given \emph{at} a particular tree in the Arboretum, the tree can be discussed in depth or used simply as a launching off point for the lecture) and give public evening lectures as part of the regular Arboretum series or those related to specific events (e.g., the Cambridge Science Festival). 
\vspace{1.5ex}\\
{\bf B. Academic year teaching in statistics \& conservation}
\vspace{1.5ex}\\
Over the next five years I will teach and develop three courses in my department. I currently teach \emph{Statistics for Biology}, a foundational statistical course for graduate and undergraduate students that covers probability theory, emphasizes the interaction between statistics and experimental design and teaches the \verb|R| programming language (team-taught with Professor John Wakeley). I also teach an undergraduate seminar course in \emph{Modern Conservation Biology}, which I designed to examine foundational and contemporary concepts and literature of conservation biology---over the next three years I plan to develop alongside this an undergraduate interactive lecture course with an emphasis on spatial and temporal ecology in conservation. % I believe that a renewed temporal ecology framework has particular relevance for conservation science in the Anthropocene, as ecological dynamics operate in increasingly nonstationary environments dominated by rising rates of anthropogenic change. 
\vspace{1.5ex}\\
{\bf C. Statistical institute: Analytical methods in modern temporal ecology}
\vspace{1.5ex}\\
A major component of my proposed program is to develop a three-week training course in temporal ecology. I believe moving forward in climate change research, as well as other areas of global change ecology where humans have modified the temporal dynamics of ecosystems, requires a major shift in the foundational training of ecologists in time, especially its unique statistical properties. Most students in ecology that I have worked with have at most used a repeated-measures ANOVA and have only a loose understanding of the common, and statistically important attributes of time-series data: their usual violation of independence assumptions because of autocorrelation, which can appear as cycles, trends and some colors of noise. Such trends also bring up the issue of stationarity (Figure 1) which is a major assumption of most statistical analyses and even most time-series approaches. 
\vspace{1.5ex}\\
As time-series are now a major type of ecological data, and are an extremely prevalent type of data in climate change ecology, the average ecologist today is routinely challenged to work with both ecological and climatological time-series. \emph{I propose to train a new generation of ecologists in time-series methods with a special emphasis on ecological and climate data and on models that can accommodate nonstationarity.} The course will include a basic introduction to time series focused on autocorrelation---cycles, trends, red and blue noise; visualization of time-series data across scales; basic classes of autoregressive models and their assumptions; modeling autocorrelation and non-stationarity; wavelet and cross-wavelet analyses; climate data: where to find it, re-use policies and important attributes (seasonal variation etc.); handling data of varying temporal scales (e.g., hourly climate data and weekly ecological data); data management of time-series data: metadata, scripting of cleaning and publication. The course will be split such that each day is a combination of interactive lectures, hands-on training in \verb|R| and time for focused data analyses. Analyses will focus mainly on developing (1) estimated climate data and best climatic predictors of phenology for the two habitats at RMBL, which will feed directly into the \emph{Phenological niches in shifting environments} research component and (2) models of recent environmental shifts with climate change and projected future change, which will feed directly into the \emph{A storage effect model for the Anthropocene} research component. 
\vspace{1.5ex}\\
The course will be team-taught. A number of colleagues are excited to assist the course including Drs. Ben Cook (climatologist from NASA, see support letter), Megan Donahue (ecologist from University of Hawaii, see support letter) and others. The course will be open to students from the early years of graduate school through to those within one year of their PhDs and priority will be given to students who will plan or are already working with nonstationary time-series data. The course will accommodate a maximum of 15 students each year, drawn from across the country following information on the course and application process being posted on the Pathways to Science database (a national database used to recruit underrepresented students), Ecolog and related listservs. 
\vspace{1.5ex}\\
I will work with Jenny Bergeron, Director of Educational Research and Assessment at Harvard's Bok Center for Teaching and Learning (see support letter) to design surveys to assess the impact of the course. These will include questions aimed to gauge students' understanding of time-series methods and concepts before and after the course, as well as the topics they found most interesting and applicable---I will use responses to these questions to guide what to teach at conference workshops on time-series methods (see D immediately below). Six months after each course students will complete a follow-up survey aimed at assessing what concepts and approaches from the course they have found most useful and a summary of `Best Practices' gained from these assessments at the end of the two courses will be posted on the course website along with all teaching materials. 
\vspace{1.5ex}\\
{\bf D. Annual workshop: Analytical methods in modern temporal ecology}
\vspace{1.5ex}\\
I plan to increase the number of students reached beyond those trained in my course by linking the course to a workshop at an annual ecological conference (expected to be the annual Ecological Society of America summer meeting). Each year 3 students from the course will be selected to lead an ESA workshop the year following their course on one topic the students found most relevant. I and a postdoctoral fellow will work with students the 3 months before the conference to organize materials and will attend the workshop as assistants both years. I will again gauge success of the course by evaluations of knowledge before and after the workshop with support from Jenny Bergeron of the Bok Center for Teaching and Learning.
\vspace{1.5ex}\\
\begin{figure}[h!]
\begin{center}
\includegraphics[width=0.8\textwidth]{/Users/Lizzie/Documents/git/grants/career/2014/figures/lilacs/lilacs.png}
\end{center}
\end{figure}

{\bf E. Lilacs \& climate change: Education and citizen science at Arnold Arboretum}
\vspace{1.5ex}\\
My lab is fortunate to be based at an institution focused both on research and on education of all sectors of the population. The Arnold Arboretum is an iconic outdoor space for residents of Boston with thousands of visitors each month. It has a tremendous diversity of outreach events through links with K-12 schools, colleges and universities and its adult education programs. The recent addition of faculty based at the Arboretum has provided a major opportunity to integrate on-site research programs with the Arboretum's established education programs. 
\vspace{1.5ex}\\
I plan to capitalize on this opportunity and meet a major educational need of the Arboretum by building a new program about climate change around lilac (\emph{Syringa vulgaris}) phenology and the Arboretum's largest annual event---Lilac Sunday. Lilac Sunday is a spring event where the Arboretum celebrates its lilac collection. The event is always held on Mother's Day in May, which previously has been during the usual period of peak bloom for the lilac collection. With climate change, however, the event now may fall well after peak bloom as lilac bloom dates march earlier and earlier. Across the US a major network of cloned lilacs established over 50 years ago has been a sentinel indicator of this march towards earlier springs with climate change \citep{Schwartz2000,Schwartz:2010nx}. Since 2007 the Arboretum has joined this program---now based through the USA National Phenology Network (with which I have worked with since 2009)---contributing data (Figure 8) but has not publicized this program and has little information for visitors on climate change. I believe the combination of this record---combined with other local records from the lilac network (Figure 8) and the public interest in Lilac Sunday---provide a major opportunity to engage with the public on climate change.
\vspace{1.5ex}\\
I will launch a new program on climate change with several major components. The first component will be an educational exhibit on climate change as seen through lilac phenology with links to impacts on wild plant species and communities---building on concepts from both major research components. This exhibit will be based at the Arboretum's Hunnewell Building Visitor Center from March to June of the three years where I will also have a major effort to reach the public on Lilac Sunday. I and one postdoctoral fellow will lead a team of undergraduates on developing materials and staffing the exhibit on Lilac Sunday. I also plan to develop a tour of the Arboretum focused on climate change that will visit various plants to understand climate variability and climate change.  All tours and exhibits will link to an initiative---called \emph{Tree Spotters}---which will encourage citizens to collect their own phenology data on a selected set of ecologically and phylogenetically diverse trees and shrubs at the Arboretum via the USA National Phenology Network's Nature Notebook (see letter of support). A postdoctoral fellow and trained undergraduates will be available to help visitors get started with Nature's Notebook each year of the exhibit. Success of the outreach program will be gauged by involvement in Nature's Notebook based at the Arboretum and through surveys before and after participation in the \emph{Tree Spotters} program, tours etc.. These surveys will again be supported by expertise provided by the Bok Center for Teaching and Learning and will focus especially on understanding how the educational and citizen science programs have altered participants' attitudes on the drivers of climate change, opportunities for mitigation and self-empowerment to gather data and otherwise affect change. This proposed work is a major undertaking but I have been piloting citizen science data collection at the Arboretum and have found with care to share tasks across postdoctoral fellows, undergraduates and volunteers is it highly feasible. 
\vspace{1.5ex}\\
I expect this work at the Arnold Arboretum will feedback to my research in major ways. Working with a number of citizen scientists over the last 5 years I have found their ideas, creativity and observations often lead me to new analyses, approaches and projects, and in explaining my research to them I more often than not have new insights. Additionally, though I propose to start these programs via my NSF CAREER award I expect to find ways to continue them long into the future. At the heart of my work is documenting trends that can only be seen on longer time-scales, and capturing temporal data that---by the very nature of climate change---cannot be re-measured or checked the next year. As such this component is central to who I am as a researcher and teacher and I believe will provide continuing opportunities for understanding the environment in an era marked by human-induced change.
% More than once I have had citizen scientists pitch me ideas that were effectively major ecological concepts---including the storage effect model and lag effects... Thus I see these as data whose value will only increase with time
\vspace{1.5ex}\\
\emph{Conclusions}
\vspace{1.5ex}\\
Continuing research on climate change has documented shifted phenologies of plants worldwide. It has shown how phenological niches may be stretched, distorted or squeezed by climate change \citep{CaraDonna2014}, but we are currently unable to answer: What does this mean to community assembly? My proposed research goals will at once make major advances in our understanding of phenology and its fundamental role in coexistence, with cascading effects for our ability to understand and predict shifts in phenology and community structure with climate change. My approach will adapt coexistence theory for the Anthropocene by redeveloping three major models---niche, neutral (random) and storage effect---for nonstationary climate dynamics. These projects are built on the efforts of postdoctoral and undergraduate researchers, students in the Statistical Institute and a public that continues to educate me on climate change and community assembly. Together I expect this research and education program will make fundamental advances in how we understand plant communities in nonstationary environments and provide a framework to predict what we may expect in the future. 
\begin{center}
\vspace{1.5ex}\\
{\sc IV. Timetable for major components}
\end{center}
\begin{center}
\begin{table}[h!]
\caption{Light shading indicates year(s) when component will occur. Both postdoctoral researchers will assist with the Lilacs \& climate change program and the Statistical Institute. Postdoctoral researcher 1 will work on the storage effect model, while postdoctoral researcher 2 will work on the phenological niches project. Undergraduate researchers will be trained for the Lilacs \& climate change and the phenological niches projects.}
\begin{tabular}{ | p{7cm} |  p{1.2 cm} | p{1.2 cm} | p{1.2 cm} | p{1.2 cm} | p{1.2 cm} | }  
\hline 
Component & Year 1 & Year 2 & Year 3 & Year 4 & Year 5 \\ \hline \hline
Storage effect research &  \cellcolor{blue!35} &\cellcolor{blue!35}  &\cellcolor{blue!35}   &  & \\ \hline
Train postdoc researcher 1  &  & \cellcolor{blue!35} & \cellcolor{blue!35}  &  & \\ \hline
Train undergraduate researchers &  &  \cellcolor{blue!35} &   \cellcolor{blue!35} &   \cellcolor{blue!35}&   \cellcolor{blue!35} \\ \hline
Phenological niches research &  &\cellcolor{blue!35} & \cellcolor{blue!35} & \cellcolor{blue!35} & \\ \hline
Statistical Institute &  & &   \cellcolor{blue!35} &  \cellcolor{blue!35} & \\ \hline
Summer conference workshop &  & & \cellcolor{blue!35}  &  & \cellcolor{blue!35}  \\ \hline
Train postdoc researcher 2 &  & &  &  \cellcolor{blue!35} &  \cellcolor{blue!35}\\ \hline
Lilacs \& climate change &  & \cellcolor{blue!35}&  \cellcolor{blue!35} &   \cellcolor{blue!35} & \\ \hline \hline
\end{tabular}
\end{table}
\end{center}

\clearpage
{\bf References cited:}
{\def\section*#1{}
\bibliography{/Users/Lizzie/Documents/EndnoteRelated/Bibtex/LizzieMainMinimal}
}

\end{document}

In the next five years to address my driving research question; (2) Improve the sub-discipline of temporal ecology to more easily recognize and handle nonstationarity in community assembly models; (3) train a new generation of ecologists in time-series methods that are applicable to both fundamental ecological research and climate change research (autocorrelation, scaling and nonstationarity); (4) public stuff

# lilacs "Syringa x chinensis 'Red Rothomagensis'"

I sort of have two versions of how to write this: (a) as shown below, or (b) start with the role of time in coexistence theory.
\begin{enumerate}
\item Coexistence theory previously anchored by two extremes
\begin{enumerate}
\item Hutchison's $n$ dimensional hypervolume
\item Randomness and related models: Hubbell's Neutral biodiversity theory
\end{enumerate}
\item Today we recognize a spectrum (expand, give a few citations), meaning modern coexistence theory is focused on understanding the forces (or just knowing when/where) that control the relative influence of each driver. Here I can say `for clarity, I use null (sensu Gotelli) to refer to the full suite of X, Y, Z, unlike some authors (c.f., Kraft).'' Or such ...
\item Tyranny of space in all this work
\item Role of time in coexistence theory (old to newer theory order):
\begin{enumerate}
\item A niche axis
\item old work on random models
\item storage effect
\item priority effects? (eh, no)
\end{enumerate}
\item Note growing interest in storage effect but much uses space or space for time substition via altered mixtures (Adler paper I think proposes it, then all of Levine's work)
\item Note: The storage effect is \emph{not} neutral---species coexist via blah, blah---but it does allow coexistence of species which would appear identical if observed outside of germination rates across time (short snapshots etc.).
\end{enumerate}


% From R3 (talks a lot about his gut feeling that pollinators don't matter much): Combine via: Species respond to environmental features (through space and time) and stochastic features (both demographic stochasticity, which is the stuff of neutral theory, and environmental stochasticity, which is the stuff of 'storage effect')
% ... In that sense, I was disappointed that the PI did not make stronger connections between the components of the proposal. If we take Chesson's coexistence theory as something akin to a more complete model including demographic and environmental stochasticity, we see that treating niche and neutral ideas as discrete alternatives is nonsensicalùthey are both occurring everywhere, but the relative importance of each changes in time and space (Adler et al. 2007 is becoming an increasingly 'classical' discussion in this regardùsee also HilleRisLambers et al. 2012). And so I was confused as to why the 'storage effect' model came last in the proposal, the niche and fitness bit came first, and the niche and neutral bit came in the middle. These are all parts of the same problem, and even if the approaches are different (experiments, theory, and observations), they should tie together much more strongly than they have. In this sense, the premise of the proposal's setup (niche vs neutral, etc) isn't clear or well developedà.if something like the storage effect model were built first, the empirical tests would flow nicely from it.

% From R4 (who also wants flower count data): How can the PI propose to examine community assembly without abundance data?

% R6: Both hypotheses ignore that fact that the timing of flowering must be tied to the timing of pollinator activity so flowering times cannot be random nor can they easily be non-overlapping given the short flight seasons in the north temperate zone. The first hypothesis also ignores beneficial effects of a critical mass of floral display for attracting pollinators. Simple field observations indicate that many plants visited by the same pollinators overlap in bloom time. It the Gothic data presented showing differences in bloom time, it would be interesting to know if the illustrated patterns held for all species across multiple years, and if the patterns were associated with drought or pollinator abundance. Regardless, I am not understanding the relationship posited between phenology and community assembly that would allow testing neutral versus niche theory.

% R7: However, I wanted to see far more details on this û how is the model constructed, which elements would be allowed to vary, what simulations would be done, and how would be model be parameterized (data is mentioned briefly, but not explained), and what are the implications of different results?

% R7: Neutral theory would predict that differences among species in phenological niches have no impact on their interactions with each other, resulting in frequency independent population growth rates (and thus, no stable coexistence). The long-term phenology data set is clearly vital in exploring the extent to which species are differentiated in their phenological niches, and how that relates to the ways in which they might indirectly interact through pollinators. However, simply identifying those niches and establishing that they differ from a null model (however provocative those results) cannot tell us how important those phenological niches are for stable coexistence (or neutral dynamics). I would have loved to see more exploration of the patterns found in the Inouye data set with follow up modeling, for example examining the implications of altering observed phenological niches on species abundances and community structure in a simulation model, to fully determine how phenological niches can contribute to coexistence.
