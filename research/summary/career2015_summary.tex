\documentclass[12pt,a4paper,oneside]{article}
\renewcommand{\baselinestretch}{1.8}
\usepackage{sectsty,setspace} 
\usepackage[top=1.00in, bottom=1.0in, left=1in, right=1in]{geometry} 
\usepackage{graphicx}
\usepackage{epstopdf}
\usepackage{amsmath,latexsym,amssymb,wasysym}
\usepackage{natbib}
\usepackage{textcomp}
\usepackage{wrapfig}

\setlength\parindent{0pt} % no indents throughout

\parskip=5pt
\pagenumbering{arabic}
\pagestyle{plain}
% squeeze space
\linespread{0.99}
\addtolength{\itemsep}{-0.05in}
\usepackage{multicol}
 

\newenvironment{smitemize}{
\begin{itemize}
  \setlength{\itemsep}{1pt}
  \setlength{\parskip}{0pt}
  \setlength{\parsep}{0pt}}
{\end{itemize}
}


\usepackage{fancyhdr}
\pagestyle{fancy}
\fancyhead[LO]{CAREER}
\fancyhead[RO]{E. M. Wolkovich}

\newcommand{\Section}[1]{\vspace{-8pt}\section{\hskip -1em.~~#1}\vspace{-3pt}} 

\begin{document}
\begin{center}
{\sc Project summary} %Title: The Role of Time in Structuring Anthropocene Plant Communities
\end{center}
\vspace{-1ex}
A major aim of modern ecology is to understand and predict the cascading effects of global change on species, communities and ecosystems. Predicting such effects, however, requires recognizing and understanding the recent transition of many ecological systems to nonstationary dynamics. Temporal nonstationarity is marked by data that show a trend over time (often amidst some cyclical or random variation), such as the increasing temperatures seen across the globe with climate change. Thus far efforts to examine the effects of climate change on ecological systems have lead to a wealth of studies finding that `biological spring' has shifted earlier in many parts of the world, with plants leafing and flowering days and weeks earlier than a century ago. Such work uses plant phenology---the timing of life-history events such as leaf budburst and first flower---to track how spring has shifted with a warming climate.
\vspace{1.5ex}\\
Yet, while plant phenology has been an excellent indicator of global climate change, recent research suggests communities are not shifting forward in time as one unit. Instead they exhibit complex patterns where some species shift earlier, some later, some remain static in time and others exhibit nonlinearities and reversals in how their phenologies have changed. Such responses highlight that phenological shifts are not predicted simply by climate. Instead, they may be mediated by species interactions and other biotic processes---making mechanistic understanding of how phenology operates within communities essential for understanding the diversity of responses. Because periods of plant flowering represent clear opportunities for species overlap basic processes for community assembly, such as priority effects or competitive displacement, may also operate on phenology. My previous work has leveraged this idea, using the special case of plant invasions to show that phenology may be critical to community structure. I now propose to build up from the special case of invasions to more general theory
\vspace{1.5ex}\\
\emph{I propose to integrate theory for community assembly with phenology and adapt current coexistence models to accommodate the temporal nonstationarity of climate change.} In particular, I will focus on adapting contemporary theory on how niche, neutral and storage effect processes govern communities to test how these processes may shift under a changing environment. My proposed research will benefit from the syngergistic support and training of postdoctoral, graduate and undergraduate researchers and be further enhanced by the unique resources available at the Arnold Arboretum to engage the local community.
\vspace{1.5ex}\\
\emph{Intellectual merit:} My proposed research will---(1) adapt niche and neutral theory to phenology while testing for how niche versus neutral mechanisms vary with climate change using a long-term dataset from Gothic, Colorado, and (2) develop a new version of the storage effect model that integrates phenology and nonstationarity in the environment to  test how coexistence mechanisms may shift over the long-term with climate change. This work will integrate phenology into major theories in ecology and provide an important advance in our understanding of community assembly. Further, by adapting these theories to nonstationary environments, as are seen with many global changes, I will provide a way to improve community-level predictions. 
\vspace{1.5ex}\\
\emph{Broader impacts:} I propose a suite of integrated teaching, training and citizen science activities that build off my interests in temporal ecology and leverage the tangible and personal nature of plant phenology---and its related wealth of long-term records. In addition to providing high-quality training to two postdoctoral researchers and a minimum of five undergraduate researchers I propose a 3-week intensive course aimed at graduate students in \emph{Analytical methods in modern temporal ecology} and two related workshops at summer conferences. Finally, I propose a major initiative to educate and engage the public in climate change research, especially related to how it has altered the phenology of individual species and reshaped plant communities; this component will build on the largest event at the Arnold Arboretum, Lilac Sunday, to reach a large and diverse audience.
\end{document}