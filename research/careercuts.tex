%%%% New stuff I am not using %%%%
Accurate predictions of plant communities are a major goal of global change ecology. Such predictions are critical to forecasting of future climate change and would assist in selecting optimal mitigation strategies, yet most efforts to date have focused on examining responses to the abiotic environment. 

%%%%%%%%%%%%%%%%%%%%%%%%%%%%%%%%%
%%% old somewhat good stuff %%% 
%%%%%%%%%%%%%%%%%%%%%%%%%%%%%%%%%

If communities are carefully structured to minimize temporal niche overlap and maximize variation in their phenologies, then rapidly shifting growing seasons may fundamentally disrupt plant communities in predictable ways. In contrast, if the phenology of plant communities appears, and each species acts as an isolated unit, then climate change may have more variable outcomes, less predictable outcomes.


\vspace{1ex}\\
Examining niche and neutral processes via the expected models will require a multivariate trait approach and a way to robustly estimate phenological overlap. A multivariate approach---examining the role of multiple traits (phenology, pollinators and height)---has special promise for understanding mechanisms operating within communities, but requires species-rich, community-level data with accompanying data on pollination mode, and quantitative traits that may mediate competition for pollinators, such as plant height. I propose to use a 40-year phenological dataset from an alpine meadow system in Gothic, Colorado \citep[Figure 4-5, see also letter of support from Dr. Inouye, and see][]{Inouye:2008gj}, which contains flowering times on over a hundred of species and will be accompanied by compiled trait information, including extremely detailed information on pollination guilds (Figure 5).


\vspace{1.5ex}\\
As this dataset represents a long-term time-series of flowering times it provides the additional opportunity to test how patterns have shifted over time. Recent studies using short-term experiments suggest environmental forces such as drought may shift drivers of community assembly from random to niche processes, suggesting the drivers of natural communities may additionally shift under global climate change. \emph{Using long-term phenology data I can test whether past climate change has altered the relative prevalence of niche and neutral processes in communities.} Preliminary analyses of the Gothic dataset suggest niche-partitioning of the growing season during the `average' climate years but uneven, possibly neutral, overlap in a particularly warm year (Figure 5). 
\vspace{1ex}\\


Thus, if we consider flowering phenology, within a community each species, each year, would be expected to have the ideal flowering time based on both abiotic and biotic factors. 


Yet while ecologists have used these data to document shifts in flowering times, there has been little work to examine how such shifts affect community assembly \citep{wolkovich:2010fee}. Because phenology is tied strongly to plant reproductive output and resources, such as pollinators and light, are generally limited, shifts in species phenology could have cascading fitness effects for entire communities; thus changes in the phenology of a few species may feedback to cause a cascade of plant community changes. This would be especially true if niche dynamics strongly govern flowering phenology \citep{gotelli1996}. In contrast, if neutral dynamics appear at play for phenology then our ability to predict future communities enters a realm of stochastic processes.


%%%%%%%%%%%%%%%%%%%%%%%%%%%%%%%%%
%%% Previous stuff that seems LESS GOOD %%% 
%%%%%%%%%%%%%%%%%%%%%%%%%%%%%%%%%

Both niche and neutral theory make predictions through time regarding the synchrony (or asynchrony) of population attributes \citep{Houlahan:2007qy,Vergnon:2009bh}. Niche theory predicts limiting similarity \citep{ABRAMS:1980dq}: species co-occurring must diverge in at least some of their traits to prevent competition for the same niche space.. However, species that use different pollinators may co-flower with low risk of pollinator competition. Neutral theory assumes `ecological equivalence,' that all species are functionally the same, and thus predicts community assembly by stochastic patterns for birth, death and migration of individuals. \citep{Hubbell:2001vo}. \emph{Thus, under niche theory, species that use the same pollinators should show limiting similarity in their flowering---with each species occupying a distinct period of the growing season (Figure 3b-c). Species that do not share pollinators may overlap in their flowering (Figure 3a,c). Neutral theory, in contrast, predicts that species' phenologies would appear randomly shuffled.}


Understanding how communities assemble and dis-assemble with climate change requires building from my previous work---which examined the direct effects of climate change and indirect effects through assembly using the special case of species invasions---to design and test fundamental theory on how time structures communities. Here I propose to build on this work by evaluating evidence, using experiments and observations, for the role of phenology in community assembly and by developing major new coexistence theory. In my first project, \emph{Fitness \& phenology}, I propose to test a fundamental assumption of the phenological niche model---specifically that flowering outside of the temporal range when a population generally flowers has detrimental fitness consequences. Next, I expand on the concept of the phenological niche within a community context: in the \emph{Niches \& neutrality} project I will test for the importance of phenology as a major component of species' temporal niches---and test whether the prevalence of niche versus neutral characteristics of communities varies with climate change. Finally, in my \emph{Storage effect} project I will extend the temporal storage effect model to include (1) phenology and (2) a nonstationary environment representative of climate change. In total my proposed work will redevelop three major areas of coexistence theory to include phenology.
\vspace{1.5ex}\\